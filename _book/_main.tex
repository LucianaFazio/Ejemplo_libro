% Options for packages loaded elsewhere
\PassOptionsToPackage{unicode}{hyperref}
\PassOptionsToPackage{hyphens}{url}
%
\documentclass[
]{book}
\usepackage{amsmath,amssymb}
\usepackage{iftex}
\ifPDFTeX
  \usepackage[T1]{fontenc}
  \usepackage[utf8]{inputenc}
  \usepackage{textcomp} % provide euro and other symbols
\else % if luatex or xetex
  \usepackage{unicode-math} % this also loads fontspec
  \defaultfontfeatures{Scale=MatchLowercase}
  \defaultfontfeatures[\rmfamily]{Ligatures=TeX,Scale=1}
\fi
\usepackage{lmodern}
\ifPDFTeX\else
  % xetex/luatex font selection
\fi
% Use upquote if available, for straight quotes in verbatim environments
\IfFileExists{upquote.sty}{\usepackage{upquote}}{}
\IfFileExists{microtype.sty}{% use microtype if available
  \usepackage[]{microtype}
  \UseMicrotypeSet[protrusion]{basicmath} % disable protrusion for tt fonts
}{}
\makeatletter
\@ifundefined{KOMAClassName}{% if non-KOMA class
  \IfFileExists{parskip.sty}{%
    \usepackage{parskip}
  }{% else
    \setlength{\parindent}{0pt}
    \setlength{\parskip}{6pt plus 2pt minus 1pt}}
}{% if KOMA class
  \KOMAoptions{parskip=half}}
\makeatother
\usepackage{xcolor}
\usepackage{longtable,booktabs,array}
\usepackage{calc} % for calculating minipage widths
% Correct order of tables after \paragraph or \subparagraph
\usepackage{etoolbox}
\makeatletter
\patchcmd\longtable{\par}{\if@noskipsec\mbox{}\fi\par}{}{}
\makeatother
% Allow footnotes in longtable head/foot
\IfFileExists{footnotehyper.sty}{\usepackage{footnotehyper}}{\usepackage{footnote}}
\makesavenoteenv{longtable}
\usepackage{graphicx}
\makeatletter
\def\maxwidth{\ifdim\Gin@nat@width>\linewidth\linewidth\else\Gin@nat@width\fi}
\def\maxheight{\ifdim\Gin@nat@height>\textheight\textheight\else\Gin@nat@height\fi}
\makeatother
% Scale images if necessary, so that they will not overflow the page
% margins by default, and it is still possible to overwrite the defaults
% using explicit options in \includegraphics[width, height, ...]{}
\setkeys{Gin}{width=\maxwidth,height=\maxheight,keepaspectratio}
% Set default figure placement to htbp
\makeatletter
\def\fps@figure{htbp}
\makeatother
\setlength{\emergencystretch}{3em} % prevent overfull lines
\providecommand{\tightlist}{%
  \setlength{\itemsep}{0pt}\setlength{\parskip}{0pt}}
\setcounter{secnumdepth}{5}
\usepackage{booktabs}
\ifLuaTeX
  \usepackage{selnolig}  % disable illegal ligatures
\fi
\usepackage[]{natbib}
\bibliographystyle{plainnat}
\IfFileExists{bookmark.sty}{\usepackage{bookmark}}{\usepackage{hyperref}}
\IfFileExists{xurl.sty}{\usepackage{xurl}}{} % add URL line breaks if available
\urlstyle{same}
\hypersetup{
  pdftitle={Esempio Libro},
  pdfauthor={LF},
  hidelinks,
  pdfcreator={LaTeX via pandoc}}

\title{Esempio Libro}
\author{LF}
\date{2023-07-17}

\begin{document}
\maketitle

{
\setcounter{tocdepth}{1}
\tableofcontents
}
\begin{figure}
\centering
\includegraphics[width=0.5\textwidth,height=\textheight]{images/logo.png}
\caption{Libro}
\end{figure}

\hypertarget{presentazione}{%
\chapter*{Presentazione}\label{presentazione}}
\addcontentsline{toc}{chapter}{Presentazione}

Questo libro è\ldots{}

\hypertarget{ringrazimenti}{%
\chapter*{Ringrazimenti}\label{ringrazimenti}}
\addcontentsline{toc}{chapter}{Ringrazimenti}

Si ringrazia\ldots{}

\hypertarget{introduzione}{%
\chapter*{Introduzione}\label{introduzione}}
\addcontentsline{toc}{chapter}{Introduzione}

Lo scopo di questo libro è..

\hypertarget{i-parte}{%
\chapter*{I parte}\label{i-parte}}
\addcontentsline{toc}{chapter}{I parte}

\hypertarget{storia}{%
\chapter{Storia}\label{storia}}

Il libro\ldots{}\footnote{per maggiori informazioni vedere \citet{Book}}

\hypertarget{antichituxe0}{%
\section{Antichità}\label{antichituxe0}}

I libro\ldots{}

\hypertarget{le-tavolette}{%
\subsection{Le tavolette}\label{le-tavolette}}

Le tavolette

\hypertarget{egiziani-e-romani}{%
\subsection{Egiziani e Romani}\label{egiziani-e-romani}}

Le civiltà \ldots.

\hypertarget{medioevo}{%
\section{Medioevo}\label{medioevo}}

Durante il Medioevo

\hypertarget{ii-parte}{%
\chapter*{II Parte}\label{ii-parte}}
\addcontentsline{toc}{chapter}{II Parte}

\hypertarget{formati-dei-libri}{%
\chapter{Formati dei libri}\label{formati-dei-libri}}

I libri a stampa\ldots{}

XX sostiene \ldots{} \citet{Illibro}

\hypertarget{parti-del-libro}{%
\section{Parti del libro}\label{parti-del-libro}}

Le parti del libro sono:
- colophon
- copertina \footnote{Inlcude: aletta, prima di copertina, quinta di copertina.}
- frontespizio

\hypertarget{iii-parte}{%
\chapter*{III parte}\label{iii-parte}}
\addcontentsline{toc}{chapter}{III parte}

\hypertarget{la-diffusione-del-libro}{%
\chapter{La diffusione del libro}\label{la-diffusione-del-libro}}

Il libro si è diffuso \ldots{}

\hypertarget{produzione-libro}{%
\chapter{Produzione libro}\label{produzione-libro}}

\begin{center}\includegraphics[width=0.3\linewidth]{images/logo} \end{center}

Il libro\ldots{}

\begin{table}

\caption{\label{tab:unnamed-chunk-2}Tabella 1}
\centering
\begin{tabular}[t]{c|c|c}
\hline
d & g & h\\
\hline
2 & 4 & 5\\
\hline
3 & 4 & 5\\
\hline
3 & 3 & 4\\
\hline
\end{tabular}
\end{table}

\begin{table}

\caption{\label{tab:unnamed-chunk-3}Tabella 2}
\centering
\begin{tabular}[t]{c|c|c}
\hline
a & b & c\\
\hline
6 & 5 & 4\\
\hline
2 & 4 & 5\\
\hline
\end{tabular}
\end{table}

Inoltre, il libro\ldots{}

Tabella 3

\begin{longtable}[]{@{}lll@{}}
\toprule\noalign{}
Produzione di libri & valore assoluto & Variazione percentuale \\
\midrule\noalign{}
\endhead
\bottomrule\noalign{}
\endlastfoot
opere pubblicate & 100.000 & 11,1 \\
copie stampate & 300.000 & 11,7 \\
editori attivi & 2.000 & -5,5 \\
\end{longtable}

\hypertarget{bibliografia}{%
\chapter{Bibliografia}\label{bibliografia}}

  \bibliography{book.bib,packages.bib}

\end{document}
